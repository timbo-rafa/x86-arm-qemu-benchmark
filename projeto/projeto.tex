\documentclass[a4paper,12pt]{article}
\usepackage{amsfonts}
\usepackage[brazil]{babel}
\usepackage[utf8]{inputenc}
\usepackage[T1]{fontenc}
\usepackage{times}
\rmfamily
\usepackage{lscape}
\usepackage{latexsym}
\usepackage{amscd}
\usepackage{epsf}
\usepackage{multicol}
\usepackage{url}
\usepackage[top=3cm,left=2cm,right=2cm,bottom=3cm]{geometry}
\linespread{1.35}
\usepackage{verbatim}

\def\candidato{Rafael Timbó Matos}
\def\orientador{Prof. Dr. Edson Borin}
\def\degr{${}^\circ$}

%%%%%%%%%%%%%
% Palavra em idioma estrangeiro
%%%%%%%%%%%%%
\newcommand{\etal}{\emph{et alli}}
\newcommand{\Software}{\emph{Software}}
\newcommand{\Hardware}{\emph{Hardware}}
\newcommand{\software}{\emph{software}}
\newcommand{\softwares}{\emph{softwares}}
\newcommand{\hardware}{\emph{hardware}}
\newcommand{\DOTNET}{.NET}
\newcommand{\VLIW}{VLIW}
\newcommand{\benchmark}{\emph{benchmark}}
\newcommand{\benchmarks}{\emph{benchmarks}}
\newcommand{\Benchmark}{\emph{Benchmark}}
\newcommand{\Benchmarks}{\emph{Benchmarks}}
\newcommand{\laptops}{\emph{laptops}}
\newcommand{\scripts}{\emph{scripts}}

\hyphenation{i-ni-ci-a-li-za-ção com-par-ti-lha-men-to cons-tru-ção cor-re-ções vir-tu-ais pro-ces-sa-do-res pro-ces-sa-dor me-lho-rar Per-nam-bu-co
a-pre-sen-ta ca-rac-te-ri-za-do pro-ce-sso in-ter-pre-ta-ção fer-ra-men-ta sis-te-mas ar-ma-ze-nar de-sen-vol-vi-men-to me-di-da}


%--------------------------------
\begin{document}

%---------------------------------------------------------------------
% Resumo em português
%---------------------------------------------------------------------
\thispagestyle{empty}
\bigskip
\bigskip
\centerline{\LARGE\bf Técnicas para Análise de Desempenho de}
\centerline{\LARGE\bf Máquinas Virtuais}
\bigskip\bigskip
\centerline{\large\bf Candidato: \candidato}
\centerline{\large\bf Orientador: \orientador}
\bigskip\bigskip
\centerline{\large Instituto de Computação}
\centerline{\large Universidade Estadual de Campinas}
\bigskip
\def\abstractname{\normalsize Resumo}

\begin{abstract}\normalsize

A análise de desempenho de processadores tem sido realizada através da execução
de conjuntos de aplicações denominados \benchmarks. Da mesma forma, a comunidade
científica tipicamente emprega o uso dos mesmos \benchmarks\ para avaliar o
desempenho de Máquinas Virtuais. Entretanto, resultados de pesquisas
recentes~\cite{divino2012wish} indicam que os \benchmarks\ tipicamente
utilizados para avaliação de processadores (SPEC CPU) podem mascarar a
sobrecarga das técnicas de emulação quando utilizados para avaliar máquinas
virtuais. Assim sendo, este trabalho propõe um levantamento de
\benchmarks\ voltados para a análise de máquinas virtuais e a construção de um
banco de dados para armazenar resultados de experimentos. Tais
\benchmarks\ possibilitarão a avaliação e o desenvolvimento de técnicas de
emulação em máquinas virtuais e o estudo de técnicas eficientes de
interpretação, considerando adequadamente a sobrecarga das técnicas de emulação.
\end{abstract}

%{\bf Palavras Chave}: Máquinas Virtuais, \benchmarks, \benchmarks\ para máquinas virtuais

%\clearpage
\section{Introdução e Justificativa}
\label{sec:intro}
% Máquinas virtuais
Uma máquina virtual é uma duplicata isolada e eficiente de uma máquina
real~\cite{popek_goldberg}. Este termo é utilizado atualmente de forma mais
abrangente, não estando necessariamente associado à máquinas físicas. Java, ou
JVM\footnote{Do inglês: \emph{Java Virtual Machine}.}, por exemplo, não está
associada a um \hardware.

No ramo de linguagens de programação, máquinas virtuais permitem a independência
de programas em relação às plataformas, possibilitando a execução em diversos
sistemas. A plataforma Java~\cite{java}, introduzida pela Sun Microsystems, e a
infraestrutura de linguagem comum~\cite{cli}, ou CLI\footnote{Do inglês:
  \emph{Common Language Infrastructure}.}, da Microsoft, são exemplos de
implementações de máquinas virtuais que proveem suporte à execução de programas
independente de plataforma. Recentemente, a empresa Google lançou mão da
tecnologia de máquinas virtuais para executar as aplicações da plataforma
Android~\cite{ehringer2010dalvik}.

A execução de múltiplos sistemas operacionais em um único processador pode ser
alcançada através da virtualização de processadores. Esta tecnologia faz uso
eficiente do processador através do compartilhamento de recursos. A VMware tem
lançado máquinas virtuais que executam múltiplos sistemas operacionas em um
mesmo processador, ou servidor, de forma segura, abrindo espaço para o
compartilhamento de recursos computacionais. A virtualização de processadores
possibilita também a execução de sistemas operacionais em processadores com
instruções distintas. A máquina virtual Virtual PC~\cite{pc} executa, em uma
plataforma Macintosh, sistemas operacionais Microsoft Windows.

Emuladores são utilizados para se executar programas que foram compilados para
uma arquitetura com um conjunto de instruções distinto. Neste âmbito, temos como
exemplo as ferramentas Rosetta~\cite{rosetta}, Digital FX!32~\cite{fx} e Intel
IA-32 EL~\cite{intel}. A primeira faz parte da distribuição do sistema
operacional Mac OS X, da Apple, permitindo aos novos sistemas (baseados na
arquitetura x86) executarem programas compilados para a arquitetura
PowerPC. Analogamente, os emuladores Intel IA-32 EL e Digital FX!32 permitem a
execução de programas baseados na arquitetura x86 em Itanium e Alpha,
respectivamente. Embora semelhantes, Smith e Nair~\cite{smith_nair_1}
classificam emuladores e ferramentas para virtualização de processadores de
forma diferente. Enquanto estas são classsificadas como máquinas virtuais de
sistema, aqueles são máquinas virtuais de processo.

A análise e instrumentação de programas binários durante a sua execução podem
ser efetuadas por ferramentas de instrumentação dinâmica de binários, como
Pin~\cite{pin} e Valgrind~\cite{val}. Neste caso, os binários são geralmente
aplicações compiladas para o mesmo conjunto de instruções, de modo que as
ferramentas servem para instrumentar estes códigos binários. Otimizadores
dinâmicos de binários, como o Dynamo~\cite{dyn}, desenvolvido pela HP, e o
StarDBT~\cite{star}, desenvolvido pela Intel, são máquinas virtuais semelhantes
às ferramentas de instrumentação, porém focam na melhora do desempenho através
da otimização, no lugar da instrumentação.

Na área de arquitetura de computadores, as máquinas virtuais possibilitam a
melhora de desempenho e a redução de energia através da otimização dinâmica de
binários. Os processadores da Transmeta - Crusoe~\cite{halfhill} e
Efficeon~\cite{Kre03} - implementam a arquitetura x86 a partir de uma máquina
virtual com projeto integrado de \software\ e \hardware. O \hardware\ implementa
um processador com instruções VLIW\footnote{Do inglês: \emph{Very-Long
    Instruction Word}.} proprietárias e o \software~\cite{soft}, executando
sobre o \hardware\ proprietário, emula a arquitetura x86 através da
interpretação e tradução dinâmica de binários. Nesta situação, uma otimização
dinâmica do código acompanha a tradução dinâmica, para melhoria de desempenho e
redução de energia.

Na área de desenvolvimento de sistemas, as máquinas virtuais permitem a execução
de testes de \software\ de grande porte sem que o \hardware\ esteja
completamente implementado. As máquinas virtuais se tornam essenciais nesta
esfera, pois a implementação do \hardware\ é concomitante ao desenvolvimento do
\software.

Todas estas máquinas virtuais podem ser definidas como programas de computador
que emulam\footnote{Smith e Nair~\cite{smith_nair_1} definem emulação como a
  implementação da interface e da funcionalidade de um (sub)sistema em outro com
  interfaces e funcionalidades diferentes.} interfaces para execução de outros
programas. A interface emulada por uma máquina virtual pode ser uma interface
concebida para auxiliar a execução de uma nova linguagem de alto nível, como é o
caso do \emph{bytecode} na máquina virtual Java, ou o conjunto de instruções da
arquitetura de um processador, como no conjunto de instruções x86 no processador
Efficeon. Muitas destas máquinas utilizam técnicas de emulação similares, como
interpretação e tradução dinâmica de binários.

A interpretação é uma técnica de emulação utilizada para execução de instruções
de uma máquina hóspede. Um interpretador executa as instruções do programa
hóspede uma a uma, em um ciclo que envolve o carregamento, a análise e a
execução da instrução. A ferramenta Bochs~\cite{boch} é um exemplo de máquina
virtual que emula a arquitetura x86 utilizando interpretação.

O processo é relativamente simples e possibilita que o interpretador seja
implementado geralmente em uma linguagem de alto nível, o que permite uma maior
portabilidade. A simplicidade do procedimento também auxilia na depuração e
validação do software. A execução de instruções uma a uma, em ordem, facilita o
tratamento de interrupções e exceções durante a execução.%NewPage

Apesar da simplicidade, este processo é geralmente lento, o que pode tornar
inviável a utilização desta técnica em algumas máquinas virtuais. Segundo Smith
e Nair~\cite{smith_nair_1}, um dos principais motivos para a interpretação ser
lenta é o fato de que a execução de cada instrução é acompanhada de uma análise
- como, por exemplo, a decodificação - e da busca e execução de uma rotina para
emular o comportamento da instrução. O interpretador da ferramenta FX!32, apesar
de ser otimizado manualmente, executa em média 45 instruções da máquina Alpha
para emular uma instrução x86~\cite{fx32}. 

A tradução dinâmica de binários consiste em traduzir dinamicamente o código da
arquitetura hóspede (sendo emulada) para a arquitetura hospedeira (nativa) e
executar este código traduzido sempre que a emulação do código for necessária.
Este processo elimina o passo de busca, decodificação e despacho da
interpretação.  A tradução dinâmica de binários envolve um custo de
inicialização associado à tradução do programa hóspede durante a execução. No
entanto, o código traduzido é geralmente armazenado em uma região de memória e
reutilizado quando o mesmo trecho do programa hóspede precisa ser reexecutado. A
reutilização do código traduzido amortiza o custo de inicialização.
% e, nos casos
%onde a execução do código é frequente, a tradução dinâmica de binários provê uma
%execução mais rápida que a interpretação. 

\subsection{\Benchmarks\ para Avaliação de Máquinas Virtuais}
\label{subsec:av}
A tradução dinâmica de binários é uma técnica eficiente para execução de código
frequente, também conhecido como código quente. Em trechos de código
frequentemente executados, o tempo de inicialização, usado pela tradução, é
apenas uma pequena fração do tempo de execução. Nesses casos, o bom desempenho
do código traduzido amortiza o custo de inicialização, o que torna a tradução
dinâmica de binários mais vantajosa do que a técnica de interpretação. Por outro
lado, para trechos de código pouco executados, ou código frio, o custo de
inicialização pode ser muito maior do que o tempo de execução do código
traduzido.

Existem diversos tipos de aplicações onde o custo de inicialização pode tornar o
uso de um tradutor de binários impraticável. Por exemplo, aplicações de tempo
real, onde tarefas devem ser completadas em um determinado limite de tempo,
podem ter suas restrições de tempo violadas por causa do tempo de inicialização
da tradução de binários. Aplicações compostas por múltiplos programas de
execução curta, onde a execução termina antes da execução do código otimizado
amortizar o custo de inicialização, podem ter o seu tempo total de execução
aumentado significativamente. O processo de inicialização do sistema
operacional, frequente em plataformas como \laptops\ e dispositivos móveis,
também é caracterizado pela execução de muito código frio, e o custo de
inicialização pode tornar a tradução dinâmica inviável. Nesses casos, combinar
interpretação com tradução dinâmica de binários é imprescindível.  De fato,
resultados de pesquisas recentes no Laboratório de Sistemas de Computação do
Instituto de Computação da Unicamp~\cite{divino2012wish} indicam que os
\benchmarks\ tipicamente utilizados para avaliação de processadores (Spec CPU)
podem mascarar a sobrecarga das técnicas de emulação quando utilizados para
avaliar máquinas virtuais. 

Apesar dos exemplos mencionados acima, a maioria dos trabalhos
relacionados~\cite{Hen00, pin} só apresenta resultados de experimentos com
\benchmarks\ caracterizados por gastar a maior parte do tempo de execução em
código quente, como o \benchmark\ Spec CPU~\cite{Hen00}. Este projeto de
pesquisa levantará e desenvolverá novos \benchmarks\ com características mais
próximas ao uso típico de um computador móvel ou de mesa, permitindo uma
avaliação mais realista do desempenho de máquinas virtuais. Estes
\benchmarks\ darão suporte a projetos correlacionados no laboratório de sistemas
de computação (LSC), onde técnicas de tradução dinâmica de binários eficientes
estão sendo desenvolvidas. Os \benchmarks\ serão executados nativamente e
emulados com máquinas virtuais e os resultados experimentais serão armazenados
em um banco de dados, que servirá de base para o cruzamento de dados e
repositório para as análises em outros projetos de pesquisa.
%
%\clearpage
%FIM DA INTRODUCAO
%
\subsection{Justificativa Quanto ao Candidato e Orientador}
\label{subsec:just}
%% Candidato %%
%
O aluno candidato deste projeto estudou no Colégio Militar de Salvador. O
candidato foi aprovado nos vestibulares de todas as universidades que prestou,
dentre elas, Universidade Federal de Pernambuco e Universidade Federal da Bahia,
sendo primeiro colocado nesta última no curso de Engenharia da Computação no ano
de 2010. Na Universidade Estadual de Campinas, obteve mais do que 80\% de
rendimento em todas suas matérias de computação, mantendo-se em oitavo colocado
na sua turma de Ciência da Computação no 1\degr\ semestre de 2012. %
%% Orientador %%
%
O orientador deste trabalho tem experiência na área de tradução dinâmica de
binários com diversas contribuições, incluindo novos algoritmos e suporte em
\hardware\ para implementação de máquinas virtuais eficientes. O pesquisador
trabalhou quatro anos como cientista pesquisador no \emph{Programming System
  Labs} da Intel, na Califórnia. Como parte do grupo BCT (\emph{Binary
  Compilation Technology}), ele desenvolveu e publicou diversas técnicas para
melhorar a tradução dinâmica de binários em diversos sistemas, incluindo novas
técnicas de formação de regiões~\cite{BWBW11,PGW+08,BWW+10}, técnicas de
paralelização dinâmica~\cite{WWB+09}, avaliação de desempenho de tradutores
dinâmicos de binários~\cite{BW08, BW09}, e outros~\cite{WHBW11, BWWA05,
  BWWA06}. Além destas publicações e outras patentes, listadas na súmula
curricular, o pesquisador teve a oportunidade de trabalhar diretamente na
infraestrutura do processador Efficeon~\cite{Kre03}\footnote{A infraestrutura do
  processador Efficeon foi licenciada pela empresa Intel.}, o estado da arte em
tecnologia de máquinas virtuais de sistema e de máquinas virtuais com o projeto
integrado de \hardware\ e \software.

\section{Materiais e Métodos}
\label{sec:mat}
Os testes de desempenho serão realizados em na plataforma ARM i.MX53 Quick Start
Board~\cite{imx53}, fabricada pela Freescale e em computadores com processadores
da família x86. A plataforma i.MX53 vem equipada com um \emph{System-on-a-Chip}
(SoC) com um processador Cortex-A8\texttrademark\ de 1GHz integrado com unidades
de processamento de vídeo, de imagem e de gráficos 2D e 3D. Além disso, a
plataforma possui 1GB de memória RAM e conexões ethernet, USB, RD 232, PATA,
SATA II, e VGA. O sistema é fornecido com um cartão SD (de 4GB) com uma
distribuição Linux Ubuntu Lucid Lynx 2.6.35.3~\cite{ubuntu}, incluindo o
compilador gcc-4.4.3~\cite{Gough_anintroduction}.  Esta plataforma será
responsável por executar \benchmarks, como o SPEC CPU~\cite{Hen00}, de forma
nativa e emulada. As máquinas virtuais responsáveis pela emulação desfrutarão de
diversas técnicas de emulação, que poderão então ser testadas.

%%%%%%%%%%%%%%%%%%%%%%%%%%%%% MAQUINAS VIRTUAIS UTILIZADAS %%%%%%%%%%%%%%%%%%%%%%%
Como exemplo de máquinas virtuais usaremos as ferramentas QEMU~\cite{qemu} e
DynamoRIO~\cite{BGA03}. O QEMU é um tradutor dinâmico de binários flexível
capaz de emular tanto um processo quanto um sistema operacional, o DynamoRIO é
um tradutor dinâmico de binários eficiente desenvolvido pelo MIT. Pretendemos
utilizar a ferramenta Bochs~\cite{boch} para os testes de interpretação.
%%%%%%%%%%%%%%%%%%%%%%%%%%%%%%%%%%%%%%%%%%%%%%%%%%%%%%%%%%%%%%%%%%%%%%%%%%%%%%%%%%

Com base em técnicas da literatura~\cite{opt_branch,klint}, os
\benchmarks\ atuais serão adaptados para ponderar também código frio, simulando
mais autenticamente uma máquina virtual. O objetivo é reproduzir com fidelidade
os métodos utilizados em aplicações reais durante a avaliação dos
interpretadores e tradutores dinâmicos de binários.

Um banco de dados será desenvolvido para armazenamento dos resultados de
experimentos, que será implementado utilizando a ferramenta
mysql~\cite{mysql2001}.  O mesmo servirá de base para consolidar e armazenar
resultados de experimentos provenientes deste e de outros projetos do
laboratório de sistemas de computação do Instituto de Computação da Unicamp.

\section{Plano de Trabalho}
\label{sec:plano}
Inicialmente, nos primeiros meses, \benchmarks\ atuais serão analisados para
servir de base para a construção de um novo \benchmark, sendo registradas as
situações de interpretação não testadas. Em seguida, novos \benchmarks\ focados
em máquinas virtuais serão desenvolvidos e testados, com posteriores correções e
otimizações. Paralelo à isto, será desenvolvido o banco de dados.
%
\subsection{Cronograma}

\begin{center}
\begin{footnotesize}
\begin{tabular}{||l||p{0.2cm}|p{0.2cm}|p{0.2cm}|p{0.2cm}|p{0.2cm}|p{0.2cm}|p{0.2cm}|p{0.2cm}|p{0.2cm}|p{0.2cm}|p{0.2cm}|p{0.24cm}||}
  \hline
  \hline
                        & \multicolumn{12}{|c||}{M\^es}\\
  \hline
  \hline
  Itens &1 &2 &3 &4 &5 &6 &7 &8 &9 &10&11&12\\
  \hline

  Implementação do banco de dados &$\bullet$&$\bullet$&$\bullet$&         &         &         &         &         &         &         &          &         \\
  \hline

  Estudo e levantamento de \benchmarks           &$\bullet$&$\bullet$&$\bullet$&$\bullet$&         &         &         &         &         &         &          &         \\
  \hline

  Testes do banco de dados        &         &         &         &$\bullet$&$\bullet$&         &         &         &         &         &          &         \\
  \hline

  Implementação dos \benchmarks   &         &         &         &$\bullet$&$\bullet$&$\bullet$&$\bullet$&         &         &         &          &         \\
  \hline

  Experimentos de desempenho com os \benchmarks          &         &         &         &         &         &         &$\bullet$&$\bullet$&$\bullet$& $\bullet$&$\bullet$&   \\
  \hline
  
  Relatório final e artigos                      &         &         &         &         &         &         &         &         &         &         &$\bullet$&$\bullet$\\
  \hline

  \hline
\end{tabular}
\label{tab:cronograma}  
\end{footnotesize}
\end{center}

\linespread{1.0}
\begin{small}
\bibliography{ref}
\bibliographystyle{abbrv}
%\bibliographystyle{plain}
\end{small}
\end{document}
